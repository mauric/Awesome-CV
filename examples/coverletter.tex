%!TEX TS-program = xelatex
%!TEX encoding = UTF-8 Unicode
% Awesome CV LaTeX Template for Cover Letter
%
% This template has been downloaded from:
% https://github.com/posquit0/Awesome-CV
%
% Authors:
% Claud D. Park <posquit0.bj@gmail.com>
% Lars Richter <mail@ayeks.de>
%
% Template license:
% CC BY-SA 4.0 (https://creativecommons.org/licenses/by-sa/4.0/)
%
%-------------------------------------------------------------------------------
% TODO's
%-------------------------------------------------------------------------------

%-------------------------------------------------------------------------------
% CONFIGURATIONS
%-------------------------------------------------------------------------------
% A4 paper size by default, use 'letterpaper' for US letter
\documentclass[11pt, a4paper]{awesome-cv}

% Configure page margins with geometry
% \geometry{left=1.4cm, top=.8cm, right=1.4cm, bottom=1.8cm, footskip=.5cm}

% Specify the location of the included fonts
\fontdir[fonts/]

% Color for highlights
% Awesome Colors: awesome-emerald, awesome-skyblue, awesome-red, awesome-pink, awesome-orange
%                 awesome-nephritis, awesome-concrete, awesome-darknight
\colorlet{awesome}{awesome-red}
% Uncomment if you would like to specify your own color
% \definecolor{awesome}{HTML}{CA63A8}

% Colors for text
% Uncomment if you would like to specify your own color
% \definecolor{darktext}{HTML}{414141}
% \definecolor{text}{HTML}{333333}
% \definecolor{graytext}{HTML}{5D5D5D}
% \definecolor{lighttext}{HTML}{999999}

% Set false if you don't want to highlight section with awesome color
\setbool{acvSectionColorHighlight}{true}

% If you would like to change the social information separator from a pipe (|) to something else
\renewcommand{\acvHeaderSocialSep}{\quad\textbar\quad}


%-------------------------------------------------------------------------------
%	PERSONAL INFORMATION
%	Comment any of the lines below if they are not required
%-------------------------------------------------------------------------------
\name{Mauricio A.}{Caceres}
\position{Elève Ingenieur{\enskip\cdotp\enskip}Mécatronique}
\address{4 rue des Archives - Rés. Univ. Kergoat Bat A Ch 203}

\mobile{+33 07 83 09 27 19} 
\email{mauric.caceres@gmail.com}
%\homepage{www.posquit0.com}
\github{mauric}
\linkedin{mauricio}
% \stackoverflow{SO-id}{SO-name}
% \twitter{@twit}
\skype{mauric\_rck@hotmail.com}
% \reddit{reddit-id}
% \extrainfo{extra informations}

%\quote{``Must be the change that you want to see in the world."}
\quote{"Objectif: travailler en équipe pour developper des competences interpersonelles "}


%-------------------------------------------------------------------------------
%	LETTER INFORMATION
%	All of the below lines must be filled out
%-------------------------------------------------------------------------------
% The company being applied to
\recipient
  {Google Inc.}
  {1600 Amphitheatre Parkway\\Mountain View, CA 94043\\  Al'attention de Monsieur RECRUTEUR }
% The date on the letter, default is the date of compilation
\letterdate{Brest, \today}
% The title of the letter
\lettertitle{Objet : lettre de motivation pour le stage Ingénieur 4 a 6 mois}
% How the letter is opened
\letteropening{Madame, Monsieur }
% How the letter is closed
%\letterclosing{Dans l'attente de votre réponse, je vous prie d’agréer, Madame, Monsieur, mes sincères salutations.}
\letterclosing{Je reste à votre disposition pour convenir d'un rendez-vous afin de vous démontrer ma motivation lors d'un entretien}


% Any enclosures with the letter
\letterenclosure[Attaché]{Curriculum Vitae}


%-------------------------------------------------------------------------------
\begin{document}

% Print the header with above personal informations
\makecvheader

% Print the footer with 3 arguments(<left>, <center>, <right>)
% Leave any of these blank if they are not needed
\makecvfooter
  {Brest, \today}
  {Mauricio Ariel Caceres~~~·~~~Lettre de Motivation}
  {}

% Print the title with above letter informations

\begin{flushright}
	Brest, \today \\
	Google Inc.\\
	1600 av. Le foret\\
	A l'attention de Monsieur BLA BLA\\
\end{flushright}
\makelettertitle

%-------------------------------------------------------------------------------
%	LETTER CONTENT
%-------------------------------------------------------------------------------
\begin{cvletter}
	
Je suis actuellement étudiant en 5ème année à l’École d'Ingénieurs de Brest et je suis à la recherche d’un stage ingénieur, intégré dans ma formation, pour une durée de 6 mois et pouvant débuter en Juillet/Août.

\lettersection{À \quad propos}
 Étant passionné par la 
 %je souhaite
 Souhaitant me spécialiser dans le secteur des télécommunications, et afin de me permettre
 de développer et mettre en pratique mes diverses connaissances techniques et théoriques,
 suivre un stage en entreprise représente une étape cruciale dans mon parcours.
 
\lettersection{Pourquoi votre entreprise}
Je suis attiré depuis tout jeune par les chiffres et j’ai toujours été curieux et autodidacte. Je souhaite maintenant orienter ma carrière vers la comptabilité et intégrer l'entreprise XXXX, leader dans son domaine, reconnu pour son sérieux et la qualité de ses produits, (Vous montrez ici que vous connaissez l'entreprise et son domaine) dans le cadre d'un stage serait une formidable opportunité pour moi pour découvrir en détail le métier de comptable et ses missions !
\lettersection{Pourquoi moi?}
Rigoureux(se) et polyvalent(e), je serai ravi(e) de pouvoir apporter mes compétences acquises au cours de mon cursus ainsi que mon profond enthousiasme à travailler à vos côtés.


\end{cvletter}


%-------------------------------------------------------------------------------
% Print the signature and enclosures with above letter informations
\makeletterclosing

\end{document}

%-------------------------------------------------------------------------------
%	NOTAS	
%------------------------------------------------------------------------------

%- Transparence :
%Tachez de vous vendre, mais ne soyez pas hypocrite, cela ne servirait à rien et pourrait vous mettre dans l'embarras lors de l'entretien ou même après.
%
%- Clarté :
%Allez droit au but dans votre lettre, inutile de paraphraser pour faire une lettre un peu plus longue. Donnez des arguments de poids, et cohérents avec votre profil, votre objectif et l'offre a laquelle vous répondez.
%
%- Pertinence et personnalisation.
%Montrez que vous connaissez l'entreprise, le domaine d'action, cela montrera que vous vous êtes renseignés, et que vous n'écrivez pas la même lettre à chacune des offres d'emplois auxquelles vous répondez.
%Faites preuve d'originalité dans votre lettre de motivation afin de vous détacher de la masse, l'impact de votre lettre sera plus important (mais sans dépasser les limites !).
%
%- Motiver et accrocher le lecteur :
%Votre lettre doit être positive et doit motiver le recruteur à vous rencontrer.
%
%- Politesse
%Soyez courtois et poli : terminez toujours votre lettre de motivation en précisant que vous restez à disposition du lecteur pour plus d'informations et en concluant par la formule de politesse standard
%Les formules de politesse à connaître :
%• « Je reste à votre entière disposition pour plus d'informations et vous prie d'agréer, Madame Monsieur, l'expression de mes salutations distinguées. »
%• « Veuillez agréer, Madame, Monsieur, mes salutations distinguées. »
%• « Je vous prie d'agréer, Monsieur, mes respectueuses et sincères salutations.. »
%• « Je vous prie de croire, Madame, Monsieur, à l'expression de mes sentiments distingués. »
%Les formules de politesses de la lettre de motivation
%
%Les petits + à connaître :
%- Evitez les phrases trop longues.
%- Votre lettre doit être cohérente avec votre CV mais pas une redite, elle doit être complémentaire.
%- Si vous décidez de mettre un titre, essayez de donner votre profil en l'orientant vers vos objectifs.
%- Ne commencez jamais votre lettre par "Je".
%- Ne soyez pas vantard.
%- N'utilisez pas les formules négatives.
%- Evitez les répétitions.
%- Ne soyez pas pompeux, inutile de multiplier les formulations trop polies.
%- Ne soyez pas misérable, on ne vous embauchera pas par pitié.
%- Ne soyez pas trop bref (5 lignes ne sont pas suffisantes).
%- Si vous tapez votre lettre, n'utilisez pas de polices excentriques. Arial ou Times seront parfaites.
%- Une lettre envoyée par la poste ou déposée au recruteur, même tapée, doit être signée.
%
%Quelques mots clefs :
%- Action
%- Adaptabilité
%- Bon relationnel
%- Pro-action
%- Anticipation
%- Autonomie
%- Croissance
%- Culture d'entreprise
%- Défi
%- Esprit d'équipe
%- Initiative
%- Positiver
%- Disponibilité
%- Actualité
%Les mots-clés les plus utilisés dans les profils emploi sur internet
%
%Erreurs souvent commises, à éviter :
%- On dit : "Je réponds à votre offre d'emploi" et non pas "je réponds à votre demande d'emploi".
%- On sollicite des informations complémentaires et non supplémentaires.
%- On ne dit pas : « Veuillez accepter, Madame, mes respectueux hommages. » Trop lourd et inadapté au domaine professionnel.
%- Vérifiez si vous vous adressez a une dame ou un monsieur, et adaptez votre lettre en fonction.
%
%
%
%1- "A l'attention de..." et non "A l'intention de"
%
%La première partie de la lettre qui mérite votre "attention" est la zone du destinataire de la lettre : vous devez indiquer le nom de l'entreprise ou organisme et ajouter le nom de la personne à qui vous adressez la lettre. Par politesse vous devez bien respecter l'orthographe du nom/prénom et vous pouvez placer au dessus du nom "A l'attention de". Parfois on se mélange les pinceaux avec l'autre expression très proche : "à l'intention de" mais celle ci s'emploie dans le cadre d'un projet ou d'une action au sein d'une phrase.
%
%[les 3 zones pour les formules de politesse d'une lettre]
%2 - Formule d'appel : "Madame, Monsieur,"
%
%Pour ouvrir votre lettre vous devez vous adresser directement à la personne ciblée, notre recommandation est de faire simple en utilisant le "Madame," ou "Monsieur," ou "Madame, Monsieur," si vous ne savez pas à qui vous écrivez précisément.
%A éviter : ne démarrez pas avec Cher Monsieur, ou Chère Madame, qui sont trop familiers ou commerciaux (à part si vous connaissez déjà bien la personne).
%Nous vous déconseillons aussi de préciser le nom de la personne concernée : bannissez ainsi par exemple "Monsieur Quifouette," (trop pompeux) et écrivez simplement "Monsieur,".
%Enfin évitez le Mademoiselle, car risqué et qui ne sert à rien, de plus les féministes demandent l'arrêt de ce terme.
%
%3 - Formules de politesse en fin de lettre
%
%Pour finir en beauté votre lettre, il faut dire au revoir et comme dans la vraie vie (IRL) vous diriez "j'ai été ravi de vous rencontrer"... par écrit il est dans les conventions d'utiliser des expressions classiques pour exprimer vos salutations... voici des exemples simples et plus travaillés des formules de politesse et des exemples de formules de politesse à ne pas reproduire !
%Attention à bien reprendre la formule d'appel qui a ouvert la lettre, si vous avez utilisé "Madame," vous devez réutiliser uniquement "Madame," dans la formule de fin de lettre.
%
%Exemples de formules de politesse simples :
%
%A vous de choisir votre style préféré parmi toutes ces formules de politesses qui expriment soit des salutations (notre recommandation), soit des sentiments (un peu vieux jeu) ou considérations.
%• « Veuillez agréer, Madame, Monsieur, mes salutations distinguées. »
%• « Je vous prie d'agréer, Madame, Monsieur, l'expression de mes sentiments distingués. »
%• « Veuillez agréer, Monsieur, l'expression de mes meilleurs sentiments. »
%• « Je vous prie d'agréer, Madame, Monsieur, l'expression de ma considération distinguée. »
%• « Je vous prie de croire, Madame, Monsieur, à l'expression de mes sentiments distingués. »
%• « Je vous prie de croire, Madame, Monsieur, à ma considération distinguée. »
%• « Je reste à votre disposition pour convenir d'un rendez-vous afin de vous démontrer ma motivation lors d'un entretien. »
%• « Si mon profil vous intéresse, rencontrons-nous. Je vous prie d'agréer, Madame, Monsieur, mes respectueuses salutations. »
%
%Exemples de formules de politesse plus travaillées :
%
%• « Dans l'attente de votre réponse, je vous prie d’agréer, Madame, Monsieur, mes sincères salutations. »
%• « Je vous prie d'agréer, Monsieur, mes respectueuses et sincères salutations. »
%• « Dans l'attente d'une réponse de votre part, je vous prie, Monsieur, Madame, de bien vouloir recevoir mes plus respectueuses salutations »
%• « Avec mes respectueux hommages, je vous prie d’agréer, Monsieur, Madame, l’expression de ma considération la plus distinguée. »
%• « Je serai heureux que ma candidature ait su vous convaincre et me permette de vous rencontrer lors d'un entretien à votre convenance. Dans cette perspective, je vous saurai gré, Monsieur, Madame, d'accepter mes respectueuses salutations. »
%
%Exemple de formules de politesse à ne pas utiliser :
%
%• « Je vous prie d'agréer, Madame, Monsieur, l'expression de mes salutations distinguées. » Explication : les salutations ne s’associent pas à l’expression. C’est l’un ou l’autre ! On exprime des sentiments et non des salutations.
%• « Je vous prie d’agréer, Madame, l’expression de mes respectueux hommages. »
%Explication : formule trop lourde et ancienne...
%• « Je vous prie de croire, Madame, Monsieur, en l'expression de mes sentiments distingués. »
%Explication : la formule qui doit être utilisée pour la lettre de motivation est "...croire à ..." et non ... "...croire en..." croire EN dieu, EN la Justice, de manière globale et limite inconditionnelle (mettre toute sa confiance dans quelque chose...) C'est donc un peu exagérer de demander cela à son recruteur :)
%Sur la même logique pour certains recruteurs : on ne met jamais des sentiments..., des trucs dévoués ou sincères...etc... dans ce type de courriers. (formules réservées dans l'ancien temps à Mamie !! )
%• « Je reste à votre entière disposition pour un éventuel entretien. En espérant que ma candidature retiendra votre attention. Cordialement, » cette phrase est à éviter car elle vous place en situation d’infériorité et n'est pas assez positive et dynamique.
%
%Autre conseil final à ne pas oublier, utilisez toujours le vouvoiement... la première marque de politesse.

