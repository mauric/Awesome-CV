%-------------------------------------------------------------------------------
%	SECTION TITLE
%-------------------------------------------------------------------------------
\cvsection{Projets}


%-------------------------------------------------------------------------------
%	CONTENT
%-------------------------------------------------------------------------------
\begin{cventries}
%---------------------------------------------------------
\cventry
{Projet Professionalisant en Equipe, SCAMI (simulation chaîne d'assemblement en milieu industriel)} % Job title
{ENIB - 5ème année} % Organization
{Brest, France} % Location
{Janvier 2015 - Juin. 2015 - 16 semaines} % Date(s)
{
	\begin{cvitems} % Description(s) of tasks/responsibilities
		\item {Mise à jour du module de Simulation d'une Chaîne d'assemblage en Milieu Industriel (SCAMI)}
		%        \item {Changement d'automate, programmation, mise en fonctionnement et supervision}
		% \item {\textbf{Référence } : Mr. Laurent Pelt Encadrant du projet}
	\end{cvitems}
}
%---------------------------------------------------------
  \cventry
    {Introduction à la Recherche en Systèmes Embarqués} % Job title
    {Faculté d'Ingénierie, Université Nationale du Cuyo - 5ème année} % Organization
    {Mendoza, Argentine} % Location
    {Août 2014 - Décembre 2014 - 20 semaines} % Date(s)
    {
      \begin{cvitems} % Description(s) of tasks/responsibilities
        \item {Mise en \oe{uvre} d'une plate-forme de développement Open Source pour une carte Raspberry Pi,interaction avec l'hardware }
      \end{cvitems}
    }
%---------------------------------------------------------
\cventry
{Conception mécanique et simulation numérique cinématique d'un robot de 6 degrés de liberté} % Job title	
{Faculté d'Ingénierie, Université Nationale du Cuyo - 5ème année} % Organization
{Mendoza, Argentine} % Location
{Janvier 2014 - Juillet 2014} % Date(s)
{
	\begin{cvitems} % Description(s) of tasks/responsibilities
	\item {Modélisation mathématique et implémentation en Matlab. Conception fait en SolidWorks}
	\end{cvitems}
}



\end{cventries}
